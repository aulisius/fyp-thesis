\chapter{Setting up LLVM}


All LLVM passes are subclasses of the Pass class, which implement functionality by overriding virtual methods inherited from Pass. Depending on how your pass works, you should inherit from the ModulePass , CallGraphSCCPass, FunctionPass , or LoopPass, or RegionPass, or BasicBlockPass classes, which gives the system more information about what your pass does, and how it can be combined with other passes. One of the main features of the LLVM Pass Framework is that it schedules passes to run in an efficient way based on the constraints that your pass meets (which are indicated by which class they derive from).

\section{Setting up the build environment}

\begin{lstlisting}
add_llvm_loadable_module( LLVMHello  
Hello.cpp

PLUGIN_TOOL
opt
)
\end{lstlisting}

\pagebreak
\section{Basic code required}

\begin{lstlisting}
#include "llvm/Pass.h"
#include "llvm/IR/Function.h"
#include "llvm/Support/raw_ostream.h"

using namespace llvm;

namespace {
struct Hello : public FunctionPass {
static char ID;
Hello() : FunctionPass(ID) {}

bool runOnFunction(Function &F) override {
errs() << "Hello: ";
errs().write_escaped(F.getName()) << '\n';
return false;
}
}; // end of struct Hello
}  // end of anonymous namespace

char Hello::ID = 0;
static RegisterPass<Hello> X("hello", "Hello World Pass",
false /* Only looks at CFG */,
false /* Analysis Pass */);
\end{lstlisting}