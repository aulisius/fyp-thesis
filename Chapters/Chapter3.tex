% Chapter 3

\chapter{REQUIREMENTS ANALYSIS} % All Chapter Headings in ALL CAPS

\section{Requirements}
\subsection{LLVM}
\textbf{Hardware Requirements}
\begin{verbatim}
1. Disk space of 15-20 GB for Debug build of LLVM and CLANG
2. Pentium or up processor
3. 32-bit ABI capable processor
4. x86 / ARM / amd64 / PowerPC architecture
\end{verbatim}
\textbf{Software Requirements}
\begin{verbatim}
1) Linux with kernel version > 4 / Cygwin on Windows
2) GNU Make	3.79, 3.79.1 - Makefile/build processor
3) GCC	>=4.8.0	C/C++ -  compiler
4) python	>=2.7 -	Automated test suite
5) zlib	>=1.2.3.4 -	Compression library 
6) Standard Linux/Unix build utilities
\end{verbatim}
\textbf{Functional Requirements for the optimization Module}
\begin{verbatim}
1) The code that is optimized must be compiled with an additional flag, one for each additional pass
2) Array access should be inside a loop, with scope for predictive commoning to be applied
3) The optimized IR representation should have lesser loads and stores, with first iteration having been hoisted out of the loop body, and the remaining loads promoted to registers
\end{verbatim}
\subsection{Analysis}
The optimization module, compiled from source at the host with matching CLANG, OPT utilities, should apply the predictive commoning to the input code. LLVM is chosen since it has a reliable class reference superior to it's compiler conterparts. The optimizing passes approach neccesitates that we write three distinct passes that accomplish predictive commoning.  we require a modern host C++ toolchain, both compiler and standard library, in order to build LLVM.
]