% Chapter 1

\chapter{INTRODUCTION} % Write in your own chapter title All Chapter headings in CAPS
\section{Objective} % All Section headings in Title Case
Low Level Virtual Machine(LLVM)\cite{Lattner:2004:LCF:977395.977673} is a compiler framework designed to support transparent, lifelong program analysis and transformation for arbitrary programs, by providing high-level information to compiler transformations at compile-time, link-time, run-time, and in idle time between runs. LLVM defines a common, low-level code representation in Static Single Assignment (SSA) form, and uses a universal Intermediate representation (IR) for representing code at a level slightly above the assembly code. LLVM has matured into a full fledged compilation ecosystem with Clang being an excellent multi-stage optimizing compiler that can be a drop in replacement for GCC. We aim to implement Predictive commoning, a kind of loop optimization, in the LLVM compiler infrastructure.

\section{Problem Statement}
In loops, array accesses are often sequential in nature. This leads to use of same value from one iteration to next. Currently, the compiler loads the  value again which is redundant. We try to optimize this step by removing the load and inserting a register swap. 

\section{Need for the system}
LLVM applies loop optimizations to the generated IR, as a part of the optimization passes run by default. They include Loop Deletion, Loop extract, Loop reduce, Loop rotate, Loop Unroll, Loop unswitch and Loop simplify. However, LLVM does not apply optimization for loops in any predictive manner. While it's GCC counterpart applies the predictive commoning, LLVM applies no such optimization. We implement second order predictive commoning to the LLVM infrastructure as a loadable module which can be invoked using a special flag.

\section{Challenges}

The main challenge of this project lies in program correctness. Any modification to the code must not result in incorrect code. For example, hoisting an array access must not result in undefined behavior. To overcome these, we employ RTTI checks and make sure we don't optimize code incorrectly.

\section{Overview of thesis}
Chapter 2 discusses the existing approaches of individual modules in greater detail and proves insight into the basics of deep learning. Chapter 3 gives the requirements analysis of the project. Chapter 4 explains the overall architecture and the design of various modules. Chapter 5 gives the implementation details of the project. Chapter 6 elaborates on the results of implementation with emphasis on comparison between the execution time of comparisons. Chapter 7 concludes the thesis and states the various extensions that can be made to the project