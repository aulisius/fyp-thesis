% Chapter 4
\chapter{SYSTEM DESIGN} % All Chapter headings in ALL CAPS

\section{Overall Architecture}
The goal of this project is to build a module that facilitates the necessary optimizations pertaining to predictive commoning. This entails a few key modules which are diagrammatically represented in Figure 4.1.
\newline

The first module deals with identifying what parts of the code will be targeted for the optimization. The LLVM API provides interfaces to identify and target code i.e.\ it helps to target only loops instead of running through the entire program. 

This overall architecture can be decomposed into a few highly cohesive modules.Each module is explained in detail in the following section.

\section{Modules}
\subsection{Identifying computation expressions}

This module is responsible for the tagging of array indexing operations. It does this by connecting the multiple instructions that make up an indexing expression and tagging them as one basic block.

It repeats the process for all the indexing expressions and adds the meta data only if the indexing expression is on the LHS.

Meta data is one of the ways by which information can be passed between passes in LLVM. Here, we use meta data about the indexing expression such as 

\begin{algorithm}
	\caption{Detecting indexed sequences}
	\begin{algorithmic}
		
		\STATE $V \leftarrow$ Empty vector of indexing expressions
		\FOR{\textbf{each} $getElementPtrInst$}
		\STATE $dominantLoad \leftarrow findDominantLoad()$
		\STATE $dominatedLoad \leftarrow findDominatedLoad()$
		\IF {$dominantLoad \neq $ null  AND $dominatedLoad \neq $ null}
		\STATE $V.push(indexed expressions)$
		\ENDIF
		\ENDFOR
		\RETURN $V$
		
	\end{algorithmic}
\end{algorithm}





