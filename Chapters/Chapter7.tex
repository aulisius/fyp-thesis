% Chapter 7

\chapter{CONCLUSION AND FUTURE WORK} % Write in your own chapter title
\section{Contribution}

Array access is the costliest memory operation in a standard program, since other accesses do not involve any calculation before accessing the memory location itself. This project implements second order predictive commoning in LLVM compiler infrastructre as a loadable module. The loadablity ensures that it does not necessitate that entire LLVM infrastructure be compiled with this module to make use of this optimization. With appropriate version of Clang, we can dynamically load this module with any version of LLVM later than 3.
\section{Future work}
The loadable module is invoked using a special flag when invoking the compiler. Second order predictive commoning can help drastically reduce the number of loads and stores, which are among the costliest memory operations. Array accesses are dominant in many calculations, including common operations like JPEG and MPEG decoding. All programs that have need for significant repetitive array accesses, may be compiled with this improvement in clang and LLVM, to take advantage of this fact. As an improvement over this project, we can eliminate the need to have a specific form of array access to be optimized, and generalize it to cover even more array access formats. 

We can also send the patch to upstream developers, and merge it to the mainline release in the next LLVM release schedule. This will enable us to get a closer code scrutiny, and will help us in identifying ways to further generalize this optimization, with the help of more experienced LLVM maintainers. 