% Chapter 7

\chapter{CONCLUSIONS} % Write in your own chapter title
\section{Contributions}
This project establishes a method to integrate the various technologies that can be used for a host of applications in various fields. The interfacing of a flagship gesture recognition device like the Kinect with a low cost microcomputer like the RPI over the internet using SSH is a relatively new combination. This particular arrangement of connecting the raspberry-pi over the internet and making it interact with the Kinect is an example of the internet of things and has not been exploited to its absolute potential. Interaction between the Arduino and the Kinect has been carried out through various platforms like the internet and bluetooth, but this has not been carried forward to the raspberry-pi. Due to various power requirements and other factors most attempts at such an integration had been unsuccessful. We have successfully integrated the Kinect and the raspberry-pi. A research paper on the above mentioned arrangement was presented at the  International Conference on Advances in Computing, Communication and Informatics. ICACCI is an IEEE conference conducted annually at different locations across the globe. Our research work was recognised at the Student Research Symposium of ICACCI-2014, Delhi. The same is now a published work of research (ISBN: 978-981-09-2229-0) in the Research Publishing Services, Singapore.The implementation of the above arrangement using latest equipment is a first of its kind. 

\section{Future Work}
The basic arrangement can be extended to newer areas of application. The set of gestures defined can also be more specific to a particular task that the robot is designed to complete. More flexibility can be incorporated by allowing the user to define gesture-response pair. For example, the user can decide that the swipeRight action will make the robot move forward. The chassis and design of the robot is simple to incorporate the basic motion but can be modified according to requirements. For example, the replacement of DC motors with servo motors to enhance the manoeuvrability of the robot.  It is possible to use the more advanced Raspberry-pi B+ version, rather than the B model, to reduce power consumption by the RPI. Moreover, with a bigger budget, the more Microsoft Kinect for Windows(Developer version) can be used instead of the one used currently. Better quality camera, with a higher megapixel value can be used for the robot. The Kinect can be programmed to receive gestures from multiple users simultaneously to perform complex tasks using the robot. Extending the same idea, in a more complex environment, multiple Kinects can be integrated to accommodate more users. Potential applications include but are not limited to defence, industrial appliances, smart home technology, technology for the differently abled individuals, adventure sports/gaming, telesurgery. 

