
Smarter and more convenient ways of user interaction are becoming a common affair in today's world. This holds true in the field of robotics especially HRI (Human Robot Interaction). This project is at the convergence of gesture recognition technology and robotics. It is envisioned that the robots of the future seamlessly interact with humans and there is no better way to achieve this than gestures. \linebreak \linebreak
Essentially, this project is aimed at setting up an arrangement that helps the user interact with a robot in a remote location through the Internet. Instead of using any traditional methods of input, a gesture-based interface is suggested to make the experience more intuitive and user friendly. The basic framework includes an input device capable of recognizing and processing gestures. These gestures are then communicated over the Internet to a robot, which follows the prescribed movement. \linebreak \linebreak
A research paper on the above mentioned arrangement was presented in the IEEE conference at the Student Research Symposium in Delhi (ICACCI-2014). The same is now a published work of research (ISBN: 978-981-09-2229-0). The implementation of the above arrangement using latest equipment is a first of its kind. 
 



