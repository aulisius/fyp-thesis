Compile time optimizations, including invariant code optimization and loop optimization are among the best candidates for improving the execution time of statically compiled programs, due their wide coverage, and untapped potential in terms of optimization. This project aims to implement one such optimization targeted at loops, known as Second Order Predictive Commoning.

Instead of implementing the optimization to a specific compiler front end for a language, we leverage the modular nature of LLVM to implement the optimization in a language neutral Intermediate Representation(IR) which can be plugged into the front end of any compiler. Second Order Predictive Commoning targets indexed array accesses inside loops, and seeks to \textsl{predict} the elements that will be used in subsequent iterations, and reuses them across iterations, eliminating redundant array accesses. We make use of the LLVM core classes for achieving this, through manipulation of the IR. Eliminating significant number of memory accesses is the main goal of this project, and this directly results in the improvement of the program execution times. The optimization is implemented in a loadable module that is triggered by using a special flag that is passed during compilation.

